\chapter{Functions}

To call a function it is needed to use sequence: \\
function id + list of arguments in parens (separated by comas).

\begin{example}[Calling a function]
\begin{lstlisting}[language=intentio]
  fun main() {
    a = int(scanln());
    println(a);
  }
\end{lstlisting}
\end{example}

To define a function it is needed to use sequence: \\
keyword fun + name + list of arguments in parens (separated by comas) + body in brakets.

\begin{example}[Defining a function]
\begin{lstlisting}[language=intentio]
  fun Identity(s) {
    s == "a" or s == "b" or s == "c"
  }
\end{lstlisting}
\end{example}

\begin{example}[Some syntactic sugar]
\begin{lstlisting}[language=intentio,mathescape=true]
  fun f(x) { x * 2; }
  fun g() { f; }

  y = g()(5); # works same as y = f(5)
  y == 10;
\end{lstlisting}
\end{example}
