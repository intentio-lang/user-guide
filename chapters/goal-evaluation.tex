\chapter{Goal evaluation}

The building of \emph{result} allows for using \emph{goal evaluation}.

\begin{example}[Some expressions results]
\begin{lstlisting}[language=intentio]
  2 < 3                 # succ 3
  3 < 2                 # fail 2
  find("a", foo_string) # succ _nr_ | fail none
\end{lstlisting}
\end{example}

The main idea of \emph{goal evaluation} is to use \emph{succ/fail} to control flow in program.

\begin{example}[Simple usage of goal evaluation]
\begin{lstlisting}[language=intentio,mathescape=true]
  if (2 < 3) {
    # this code will be evaluated
  }

  if (3 < 2) {
    # this code won't be evaluated
  }

  while(find("a", foo_string)){
    pos = find("a", foo_string;
    a = cut(0,pos, foo_string);
    # this code will be evaluated 
    # as long as foo_string contains an "a"
  }
\end{lstlisting}
\end{example}

\clearpage
The type and the value are not important. It allows to use complicated conditions, combined with basic expresssions.

\begin{example}[The combined condition]
\begin{lstlisting}[language=intentio,mathescape=true]
  if (
        -30 < a < 30
        and if(a < 0){
            is_prime_num(- a)
          }else{
            is_prime_num(a)
          }
      ) {
    println(a);
  }
\end{lstlisting}
\end{example}
