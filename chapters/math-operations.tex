\chapter{Operations}

Intentio provides three types of oparators:
\begin{itemize}
  \item simple math operators
  \item relational operators
  \item logical operators
\end{itemize}

\begin{example}[Simple math operators]
\begin{lstlisting}[language=intentio]
  1 + 2       # succ 3
  1 - 2       # succ -1
  1 * 2       # succ 2
  1 / 2       # succ 0
  1 / 2.0     # succ 0.5
\end{lstlisting}
\end{example}

\begin{example}[Usage of math operators]
\begin{lstlisting}[language=intentio]
  (2 + 7) * 3 - 8 / 4 # succ 25
\end{lstlisting}
\end{example}

\clearpage
\begin{example}[Relational operators]
\begin{lstlisting}[language=intentio]
  1 < 2       # succ 2
  1 > 2       # fail 2
  1 <= 2      # succ 2
  1 >= 2      # fail 2
  1 == 2      # fail 2
  # (below) authomatic conversion from Int to Float
  1 == 1.0    # succ 1.0
  1 =! 2      # succ 2
  1 === 1     # succ 1
  1 === 1.0   # fail 1.0
  1 ==! 1     # fail 1
\end{lstlisting}
\end{example}

\begin{example}[Usage of relational operators]
\begin{lstlisting}[language=intentio]
  a = 3;
  b = 7;
  1 < a < b < 20 # succ 20
\end{lstlisting}
\end{example}

\begin{example}[Logical operators]
\begin{lstlisting}[language=intentio]
  a = succ "Ala"   
  b = fail 8

  a or b            # succ 8
  a or succ none    # succ none
  a and b           # fail 8
  succ none and a   # succ "Ala"
  a xor b           # succ 8
  a xor fail none   # succ none
  not a             # fail "Ala"
\end{lstlisting}
\end{example}
